\batchmode
\makeatletter
\def\input@path{{"D:/Hesam/Data Science/My Summaries - Cheat Sheets/SQL-versus-Pandas-master/"}}
\makeatother
\documentclass[11pt]{article}
\usepackage[T1]{fontenc}
\usepackage[latin9]{inputenc}
\usepackage{geometry}
\geometry{verbose,tmargin=1in,bmargin=1in,lmargin=1in,rmargin=1in}
\usepackage{fancyhdr}
\pagestyle{fancy}
\usepackage{color}
\usepackage{longtable}
\usepackage{booktabs}
\usepackage{textcomp}
\usepackage{amsmath}
\usepackage{amssymb}
\usepackage{graphicx}
\usepackage[pdftex,unicode=true,
 bookmarks=true,bookmarksnumbered=false,bookmarksopen=false,
 breaklinks=true,pdfborder={0 0 1},backref=section,colorlinks=true]
 {hyperref}
\hypersetup{pdftitle={How to use Pandas for SQL_like actions?},
 pdfauthor={Hesam Mohammad Hosseini},
 pdfkeywords={Pandas, SQL, Python},
 urlcolor=urlcolor,linkcolor=linkcolor,citecolor=citecolor}

\makeatletter

%%%%%%%%%%%%%%%%%%%%%%%%%%%%%% LyX specific LaTeX commands.
\newcommand*\LyXZeroWidthSpace{\hspace{0pt}}
%% Because html converters don't know tabularnewline
\providecommand{\tabularnewline}{\\}

%%%%%%%%%%%%%%%%%%%%%%%%%%%%%% User specified LaTeX commands.


    \usepackage[breakable]{tcolorbox}
\usepackage{parskip}% Stop auto-indenting (to mimic markdown behaviour)
    
    \usepackage{iftex}
\ifPDFTeX
    	\else
    	\usepackage{fontspec}
\fi

    % Basic figure setup, for now with no caption control since it's done
    % automatically by Pandoc (which extracts ![](path) syntax from Markdown).
    % Maintain compatibility with old templates. Remove in nbconvert 6.0
    \let\Oldincludegraphics\includegraphics
    % Ensure that by default, figures have no caption (until we provide a
    % proper Figure object with a Caption API and a way to capture that
    % in the conversion process - todo).
    \usepackage{caption}
\DeclareCaptionFormat{nocaption}{}
    \captionsetup{format=nocaption,aboveskip=0pt,belowskip=0pt}

    \usepackage{float}
\floatplacement{figure}{H} % forces figures to be placed at the correct location
    \usepackage{xcolor}% Allow colors to be defined
    \usepackage{enumerate}% Needed for markdown enumerations to work
    % Used to adjust the document margins
    % Equations
    % Equations
    \usepackage{textcomp}% defines textquotesingle
    % Hack from http://tex.stackexchange.com/a/47451/13684:
    \AtBeginDocument{%
        \def\PYZsq{\textquotesingle}% Upright quotes in Pygmentized code
    }
    \usepackage{upquote}% Upright quotes for verbatim code
    \usepackage{eurosym}% defines \euro
    \usepackage[mathletters]{ucs}% Extended unicode (utf-8) support
    \usepackage{fancyvrb}% verbatim replacement that allows latex
    \usepackage{grffile}% extends the file name processing of package graphics 
                         % to support a larger range
     % fix for old versions of grffile with XeLaTeX
    \@ifpackagelater{grffile}{2019/11/01}
    {
      % Do nothing on new versions
    }
    {
      \def\Gread@@xetex#1{%
        \IfFileExists{"\Gin@base".bb}%
        {\Gread@eps{\Gin@base.bb}}%
        {\Gread@@xetex@aux#1}%
      }
    }
    
    \usepackage[Export]{adjustbox}% Used to constrain images to a maximum size
    \adjustboxset{max size={0.9\linewidth}{0.9\paperheight}}

    % The hyperref package gives us a pdf with properly built
    % internal navigation ('pdf bookmarks' for the table of contents,
    % internal cross-reference links, web links for URLs, etc.)
    % The default LaTeX title has an obnoxious amount of whitespace. By default,
    % titling removes some of it. It also provides customization options.
    \usepackage{titling}
% longtable support required by pandoc >1.10
    % table support for pandoc > 1.12.2
    \usepackage[inline]{enumitem}% IRkernel/repr support (it uses the enumerate* environment)
    \usepackage[normalem]{ulem}% ulem is needed to support strikethroughs (\sout)
                                % normalem makes italics be italics, not underlines
    \usepackage{mathrsfs}

    \usepackage[printwatermark]{xwatermark}
    \usepackage{graphicx}
    \usepackage{lipsum}


\fancyhf{}               % Clear fancy header/footer
\fancyhead{} % clear all header fields

\pagebreak\LyXZeroWidthSpace\fancyhf{}               % Clear fancy header/footer 
\fancyhead[L]{Hessam Mohammad Hosseini}   % My name in Left footer 
\fancyhead[R]{Pandas for SQL users}  % {\thepage} Page number in Right footer 
%\fancyfoot[L]{Hessam Mohammad Hosseini}   % My name in Left footer 
%\fancyfoot[R]{Pandas for SQL users}  % {\thepage} Page number in Right footer 
\makeatletter \let\ps@plain\ps@fancy   % Plain page style = fancy page style \makeatother

%\fancyhead{} % clear all header fields 
\fancyhead[RO,LE]{\textbf{Pandas for SQL users}} 
\fancyfoot{} % clear all footer fields 
\fancyfoot[LE,RO]{\thepage} 
\fancyfoot[LO,CE]{\href{https://www.linkedin.com/in/hesam-mohammad-hosseini/}{Hessam Mohammad Hosseini}}
\fancyfoot[CO,RE]{\href{https://github.com/jupihes}{jupihes@github}} 
\renewcommand{\headrulewidth}{0.4pt} 
\renewcommand{\footrulewidth}{0.4pt}



%%\fancyhead[RO,LE]{\textbf{Pandas for SQL users}} 
%\fancyhead[L]{Hessam Mohammad Hosseini}   % My name in Left footer 
%\fancyhead[R]{\textbf{Pandas for SQL users}}  % {\thepage} Page number in Right footer 
%%\fancyfoot[L]{Hessam Mohammad Hosseini}   % My name in Left footer 
%%\fancyfoot[R]{MTN Irancell - ICDC2023}  % {\thepage} Page number in Right footer 
%\makeatletter \let\ps@plain\ps@fancy   % Plain page style = fancy page style \makeatother
%
%
%\fancyfoot{} % clear all footer fields 
%\fancyfoot[LE,RO]{\thepage} 
%\fancyfoot[LO,CE]{Hessam Mohammad Hosseini} 
%\fancyfoot[CO,RE]{https://github.com/jupihes}
%%\fancyfoot[L]{Hessam Mohammad Hosseini}   % My name in Left footer
%%\fancyfoot[R]{Pandas for SQL users}  % {\thepage} Page number in Right footer
%
%\renewcommand{\headrulewidth}{0.4pt} 
%\renewcommand{\footrulewidth}{0.4pt}

\makeatletter
\let\ps@plain\ps@fancy   % Plain page style = fancy page style
\makeatother

    
    % Colors for the hyperref package
    \definecolor{urlcolor}{rgb}{0,.145,.698}
    \definecolor{linkcolor}{rgb}{.71,0.21,0.01}
    \definecolor{citecolor}{rgb}{.71,0.21,0.01}%{.12,.54,.11}

    % ANSI colors
    \definecolor{ansi-black}{HTML}{3E424D}
    \definecolor{ansi-black-intense}{HTML}{282C36}
    \definecolor{ansi-red}{HTML}{E75C58}
    \definecolor{ansi-red-intense}{HTML}{B22B31}
    \definecolor{ansi-green}{HTML}{00A250}
    \definecolor{ansi-green-intense}{HTML}{007427}
    \definecolor{ansi-yellow}{HTML}{DDB62B}
    \definecolor{ansi-yellow-intense}{HTML}{B27D12}
    \definecolor{ansi-blue}{HTML}{208FFB}
    \definecolor{ansi-blue-intense}{HTML}{0065CA}
    \definecolor{ansi-magenta}{HTML}{D160C4}
    \definecolor{ansi-magenta-intense}{HTML}{A03196}
    \definecolor{ansi-cyan}{HTML}{60C6C8}
    \definecolor{ansi-cyan-intense}{HTML}{258F8F}
    \definecolor{ansi-white}{HTML}{C5C1B4}
    \definecolor{ansi-white-intense}{HTML}{A1A6B2}
    \definecolor{ansi-default-inverse-fg}{HTML}{FFFFFF}
    \definecolor{ansi-default-inverse-bg}{HTML}{000000}

    % common color for the border for error outputs.
    \definecolor{outerrorbackground}{HTML}{FFDFDF}

    % commands and environments needed by pandoc snippets
    % extracted from the output of `pandoc -s`
    \providecommand{\tightlist}{%
      \setlength{\itemsep}{0pt}\setlength{\parskip}{0pt}}
    \DefineVerbatimEnvironment{Highlighting}{Verbatim}{commandchars=\\\{\}}
    % Add ',fontsize=\small' for more characters per line
    \newenvironment{Shaded}{}{}
    \newcommand{\KeywordTok}[1]{\textcolor[rgb]{0.00,0.44,0.13}{\textbf{{#1}}}}
    \newcommand{\DataTypeTok}[1]{\textcolor[rgb]{0.56,0.13,0.00}{{#1}}}
    \newcommand{\DecValTok}[1]{\textcolor[rgb]{0.25,0.63,0.44}{{#1}}}
    \newcommand{\BaseNTok}[1]{\textcolor[rgb]{0.25,0.63,0.44}{{#1}}}
    \newcommand{\FloatTok}[1]{\textcolor[rgb]{0.25,0.63,0.44}{{#1}}}
    \newcommand{\CharTok}[1]{\textcolor[rgb]{0.25,0.44,0.63}{{#1}}}
    \newcommand{\StringTok}[1]{\textcolor[rgb]{0.25,0.44,0.63}{{#1}}}
    \newcommand{\CommentTok}[1]{\textcolor[rgb]{0.38,0.63,0.69}{\textit{{#1}}}}
    \newcommand{\OtherTok}[1]{\textcolor[rgb]{0.00,0.44,0.13}{{#1}}}
    \newcommand{\AlertTok}[1]{\textcolor[rgb]{1.00,0.00,0.00}{\textbf{{#1}}}}
    \newcommand{\FunctionTok}[1]{\textcolor[rgb]{0.02,0.16,0.49}{{#1}}}
    \newcommand{\RegionMarkerTok}[1]{{#1}}
    \newcommand{\ErrorTok}[1]{\textcolor[rgb]{1.00,0.00,0.00}{\textbf{{#1}}}}
    \newcommand{\NormalTok}[1]{{#1}}
    
    % Additional commands for more recent versions of Pandoc
    \newcommand{\ConstantTok}[1]{\textcolor[rgb]{0.53,0.00,0.00}{{#1}}}
    \newcommand{\SpecialCharTok}[1]{\textcolor[rgb]{0.25,0.44,0.63}{{#1}}}
    \newcommand{\VerbatimStringTok}[1]{\textcolor[rgb]{0.25,0.44,0.63}{{#1}}}
    \newcommand{\SpecialStringTok}[1]{\textcolor[rgb]{0.73,0.40,0.53}{{#1}}}
    \newcommand{\ImportTok}[1]{{#1}}
    \newcommand{\DocumentationTok}[1]{\textcolor[rgb]{0.73,0.13,0.13}{\textit{{#1}}}}
    \newcommand{\AnnotationTok}[1]{\textcolor[rgb]{0.38,0.63,0.69}{\textbf{\textit{{#1}}}}}
    \newcommand{\CommentVarTok}[1]{\textcolor[rgb]{0.38,0.63,0.69}{\textbf{\textit{{#1}}}}}
    \newcommand{\VariableTok}[1]{\textcolor[rgb]{0.10,0.09,0.49}{{#1}}}
    \newcommand{\ControlFlowTok}[1]{\textcolor[rgb]{0.00,0.44,0.13}{\textbf{{#1}}}}
    \newcommand{\OperatorTok}[1]{\textcolor[rgb]{0.40,0.40,0.40}{{#1}}}
    \newcommand{\BuiltInTok}[1]{{#1}}
    \newcommand{\ExtensionTok}[1]{{#1}}
    \newcommand{\PreprocessorTok}[1]{\textcolor[rgb]{0.74,0.48,0.00}{{#1}}}
    \newcommand{\AttributeTok}[1]{\textcolor[rgb]{0.49,0.56,0.16}{{#1}}}
    \newcommand{\InformationTok}[1]{\textcolor[rgb]{0.38,0.63,0.69}{\textbf{\textit{{#1}}}}}
    \newcommand{\WarningTok}[1]{\textcolor[rgb]{0.38,0.63,0.69}{\textbf{\textit{{#1}}}}}
    
    
    % Define a nice break command that doesn't care if a line doesn't already
    % exist.
    \def\br{\hspace*{\fill} \\* }
    % Math Jax compatibility definitions
    \def\gt{>}
    \def\lt{<}
    \let\Oldtex\TeX
    \let\Oldlatex\LaTeX
    \renewcommand{\TeX}{\textrm{\Oldtex}}
    \renewcommand{\LaTeX}{\textrm{\Oldlatex}}
    % Document parameters
    % Document title
    \title{Pandas SQL-like functionality}

%\newwatermark[allpages,color=red!50,angle=45,scale=1,xpos=0,ypos=0]{Python for Telecom \n Hesam Mohammad hosseini}
    
    
    
    
    
% Pygments definitions

\def\PY@reset{\let\PY@it=\relax \let\PY@bf=\relax%
    \let\PY@ul=\relax \let\PY@tc=\relax%
    \let\PY@bc=\relax \let\PY@ff=\relax}
\def\PY@tok#1{\csname PY@tok@#1\endcsname}
\def\PY@toks#1+{\ifx\relax#1\empty\else%
    \PY@tok{#1}\expandafter\PY@toks\fi}
\def\PY@do#1{\PY@bc{\PY@tc{\PY@ul{%
    \PY@it{\PY@bf{\PY@ff{#1}}}}}}}
\def\PY#1#2{\PY@reset\PY@toks#1+\relax+\PY@do{#2}}

\expandafter\def\csname PY@tok@w\endcsname{\def\PY@tc##1{\textcolor[rgb]{0.73,0.73,0.73}{##1}}}
\expandafter\def\csname PY@tok@c\endcsname{\let\PY@it=\textit\def\PY@tc##1{\textcolor[rgb]{0.25,0.50,0.50}{##1}}}
\expandafter\def\csname PY@tok@cp\endcsname{\def\PY@tc##1{\textcolor[rgb]{0.74,0.48,0.00}{##1}}}
\expandafter\def\csname PY@tok@k\endcsname{\let\PY@bf=\textbf\def\PY@tc##1{\textcolor[rgb]{0.00,0.50,0.00}{##1}}}
\expandafter\def\csname PY@tok@kp\endcsname{\def\PY@tc##1{\textcolor[rgb]{0.00,0.50,0.00}{##1}}}
\expandafter\def\csname PY@tok@kt\endcsname{\def\PY@tc##1{\textcolor[rgb]{0.69,0.00,0.25}{##1}}}
\expandafter\def\csname PY@tok@o\endcsname{\def\PY@tc##1{\textcolor[rgb]{0.40,0.40,0.40}{##1}}}
\expandafter\def\csname PY@tok@ow\endcsname{\let\PY@bf=\textbf\def\PY@tc##1{\textcolor[rgb]{0.67,0.13,1.00}{##1}}}
\expandafter\def\csname PY@tok@nb\endcsname{\def\PY@tc##1{\textcolor[rgb]{0.00,0.50,0.00}{##1}}}
\expandafter\def\csname PY@tok@nf\endcsname{\def\PY@tc##1{\textcolor[rgb]{0.00,0.00,1.00}{##1}}}
\expandafter\def\csname PY@tok@nc\endcsname{\let\PY@bf=\textbf\def\PY@tc##1{\textcolor[rgb]{0.00,0.00,1.00}{##1}}}
\expandafter\def\csname PY@tok@nn\endcsname{\let\PY@bf=\textbf\def\PY@tc##1{\textcolor[rgb]{0.00,0.00,1.00}{##1}}}
\expandafter\def\csname PY@tok@ne\endcsname{\let\PY@bf=\textbf\def\PY@tc##1{\textcolor[rgb]{0.82,0.25,0.23}{##1}}}
\expandafter\def\csname PY@tok@nv\endcsname{\def\PY@tc##1{\textcolor[rgb]{0.10,0.09,0.49}{##1}}}
\expandafter\def\csname PY@tok@no\endcsname{\def\PY@tc##1{\textcolor[rgb]{0.53,0.00,0.00}{##1}}}
\expandafter\def\csname PY@tok@nl\endcsname{\def\PY@tc##1{\textcolor[rgb]{0.63,0.63,0.00}{##1}}}
\expandafter\def\csname PY@tok@ni\endcsname{\let\PY@bf=\textbf\def\PY@tc##1{\textcolor[rgb]{0.60,0.60,0.60}{##1}}}
\expandafter\def\csname PY@tok@na\endcsname{\def\PY@tc##1{\textcolor[rgb]{0.49,0.56,0.16}{##1}}}
\expandafter\def\csname PY@tok@nt\endcsname{\let\PY@bf=\textbf\def\PY@tc##1{\textcolor[rgb]{0.00,0.50,0.00}{##1}}}
\expandafter\def\csname PY@tok@nd\endcsname{\def\PY@tc##1{\textcolor[rgb]{0.67,0.13,1.00}{##1}}}
\expandafter\def\csname PY@tok@s\endcsname{\def\PY@tc##1{\textcolor[rgb]{0.73,0.13,0.13}{##1}}}
\expandafter\def\csname PY@tok@sd\endcsname{\let\PY@it=\textit\def\PY@tc##1{\textcolor[rgb]{0.73,0.13,0.13}{##1}}}
\expandafter\def\csname PY@tok@si\endcsname{\let\PY@bf=\textbf\def\PY@tc##1{\textcolor[rgb]{0.73,0.40,0.53}{##1}}}
\expandafter\def\csname PY@tok@se\endcsname{\let\PY@bf=\textbf\def\PY@tc##1{\textcolor[rgb]{0.73,0.40,0.13}{##1}}}
\expandafter\def\csname PY@tok@sr\endcsname{\def\PY@tc##1{\textcolor[rgb]{0.73,0.40,0.53}{##1}}}
\expandafter\def\csname PY@tok@ss\endcsname{\def\PY@tc##1{\textcolor[rgb]{0.10,0.09,0.49}{##1}}}
\expandafter\def\csname PY@tok@sx\endcsname{\def\PY@tc##1{\textcolor[rgb]{0.00,0.50,0.00}{##1}}}
\expandafter\def\csname PY@tok@m\endcsname{\def\PY@tc##1{\textcolor[rgb]{0.40,0.40,0.40}{##1}}}
\expandafter\def\csname PY@tok@gh\endcsname{\let\PY@bf=\textbf\def\PY@tc##1{\textcolor[rgb]{0.00,0.00,0.50}{##1}}}
\expandafter\def\csname PY@tok@gu\endcsname{\let\PY@bf=\textbf\def\PY@tc##1{\textcolor[rgb]{0.50,0.00,0.50}{##1}}}
\expandafter\def\csname PY@tok@gd\endcsname{\def\PY@tc##1{\textcolor[rgb]{0.63,0.00,0.00}{##1}}}
\expandafter\def\csname PY@tok@gi\endcsname{\def\PY@tc##1{\textcolor[rgb]{0.00,0.63,0.00}{##1}}}
\expandafter\def\csname PY@tok@gr\endcsname{\def\PY@tc##1{\textcolor[rgb]{1.00,0.00,0.00}{##1}}}
\expandafter\def\csname PY@tok@ge\endcsname{\let\PY@it=\textit}
\expandafter\def\csname PY@tok@gs\endcsname{\let\PY@bf=\textbf}
\expandafter\def\csname PY@tok@gp\endcsname{\let\PY@bf=\textbf\def\PY@tc##1{\textcolor[rgb]{0.00,0.00,0.50}{##1}}}
\expandafter\def\csname PY@tok@go\endcsname{\def\PY@tc##1{\textcolor[rgb]{0.53,0.53,0.53}{##1}}}
\expandafter\def\csname PY@tok@gt\endcsname{\def\PY@tc##1{\textcolor[rgb]{0.00,0.27,0.87}{##1}}}
\expandafter\def\csname PY@tok@err\endcsname{\def\PY@bc##1{\setlength{\fboxsep}{0pt}\fcolorbox[rgb]{1.00,0.00,0.00}{1,1,1}{\strut ##1}}}
\expandafter\def\csname PY@tok@kc\endcsname{\let\PY@bf=\textbf\def\PY@tc##1{\textcolor[rgb]{0.00,0.50,0.00}{##1}}}
\expandafter\def\csname PY@tok@kd\endcsname{\let\PY@bf=\textbf\def\PY@tc##1{\textcolor[rgb]{0.00,0.50,0.00}{##1}}}
\expandafter\def\csname PY@tok@kn\endcsname{\let\PY@bf=\textbf\def\PY@tc##1{\textcolor[rgb]{0.00,0.50,0.00}{##1}}}
\expandafter\def\csname PY@tok@kr\endcsname{\let\PY@bf=\textbf\def\PY@tc##1{\textcolor[rgb]{0.00,0.50,0.00}{##1}}}
\expandafter\def\csname PY@tok@bp\endcsname{\def\PY@tc##1{\textcolor[rgb]{0.00,0.50,0.00}{##1}}}
\expandafter\def\csname PY@tok@fm\endcsname{\def\PY@tc##1{\textcolor[rgb]{0.00,0.00,1.00}{##1}}}
\expandafter\def\csname PY@tok@vc\endcsname{\def\PY@tc##1{\textcolor[rgb]{0.10,0.09,0.49}{##1}}}
\expandafter\def\csname PY@tok@vg\endcsname{\def\PY@tc##1{\textcolor[rgb]{0.10,0.09,0.49}{##1}}}
\expandafter\def\csname PY@tok@vi\endcsname{\def\PY@tc##1{\textcolor[rgb]{0.10,0.09,0.49}{##1}}}
\expandafter\def\csname PY@tok@vm\endcsname{\def\PY@tc##1{\textcolor[rgb]{0.10,0.09,0.49}{##1}}}
\expandafter\def\csname PY@tok@sa\endcsname{\def\PY@tc##1{\textcolor[rgb]{0.73,0.13,0.13}{##1}}}
\expandafter\def\csname PY@tok@sb\endcsname{\def\PY@tc##1{\textcolor[rgb]{0.73,0.13,0.13}{##1}}}
\expandafter\def\csname PY@tok@sc\endcsname{\def\PY@tc##1{\textcolor[rgb]{0.73,0.13,0.13}{##1}}}
\expandafter\def\csname PY@tok@dl\endcsname{\def\PY@tc##1{\textcolor[rgb]{0.73,0.13,0.13}{##1}}}
\expandafter\def\csname PY@tok@s2\endcsname{\def\PY@tc##1{\textcolor[rgb]{0.73,0.13,0.13}{##1}}}
\expandafter\def\csname PY@tok@sh\endcsname{\def\PY@tc##1{\textcolor[rgb]{0.73,0.13,0.13}{##1}}}
\expandafter\def\csname PY@tok@s1\endcsname{\def\PY@tc##1{\textcolor[rgb]{0.73,0.13,0.13}{##1}}}
\expandafter\def\csname PY@tok@mb\endcsname{\def\PY@tc##1{\textcolor[rgb]{0.40,0.40,0.40}{##1}}}
\expandafter\def\csname PY@tok@mf\endcsname{\def\PY@tc##1{\textcolor[rgb]{0.40,0.40,0.40}{##1}}}
\expandafter\def\csname PY@tok@mh\endcsname{\def\PY@tc##1{\textcolor[rgb]{0.40,0.40,0.40}{##1}}}
\expandafter\def\csname PY@tok@mi\endcsname{\def\PY@tc##1{\textcolor[rgb]{0.40,0.40,0.40}{##1}}}
\expandafter\def\csname PY@tok@il\endcsname{\def\PY@tc##1{\textcolor[rgb]{0.40,0.40,0.40}{##1}}}
\expandafter\def\csname PY@tok@mo\endcsname{\def\PY@tc##1{\textcolor[rgb]{0.40,0.40,0.40}{##1}}}
\expandafter\def\csname PY@tok@ch\endcsname{\let\PY@it=\textit\def\PY@tc##1{\textcolor[rgb]{0.25,0.50,0.50}{##1}}}
\expandafter\def\csname PY@tok@cm\endcsname{\let\PY@it=\textit\def\PY@tc##1{\textcolor[rgb]{0.25,0.50,0.50}{##1}}}
\expandafter\def\csname PY@tok@cpf\endcsname{\let\PY@it=\textit\def\PY@tc##1{\textcolor[rgb]{0.25,0.50,0.50}{##1}}}
\expandafter\def\csname PY@tok@c1\endcsname{\let\PY@it=\textit\def\PY@tc##1{\textcolor[rgb]{0.25,0.50,0.50}{##1}}}
\expandafter\def\csname PY@tok@cs\endcsname{\let\PY@it=\textit\def\PY@tc##1{\textcolor[rgb]{0.25,0.50,0.50}{##1}}}

\def\PYZbs{\char`\\}
\def\PYZus{\char`\_}
\def\PYZob{\char`\{}
\def\PYZcb{\char`\}}
\def\PYZca{\char`\^}
\def\PYZam{\char`\&}
\def\PYZlt{\char`\<}
\def\PYZgt{\char`\>}
\def\PYZsh{\char`\#}
\def\PYZpc{\char`\%}
\def\PYZdl{\char`\$}
\def\PYZhy{\char`\-}
\def\PYZsq{\char`\'}
\def\PYZdq{\char`\"}
\def\PYZti{\char`\~}
% for compatibility with earlier versions
\def\PYZat{@}
\def\PYZlb{[}
\def\PYZrb{]}



    % For linebreaks inside Verbatim environment from package fancyvrb. 
    
        \newbox\Wrappedcontinuationbox 
        \newbox\Wrappedvisiblespacebox 
        \newcommand*{\Wrappedvisiblespace}{\textcolor{red}{\textvisiblespace}} 
        \newcommand*{\Wrappedcontinuationsymbol}{\textcolor{red}{\llap{\tiny$\m@th\hookrightarrow$}}} 
        \newcommand*{\Wrappedcontinuationindent}{3ex } 
        \newcommand*{\Wrappedafterbreak}{\kern\Wrappedcontinuationindent\copy\Wrappedcontinuationbox} 
        % Take advantage of the already applied Pygments mark-up to insert 
        % potential linebreaks for TeX processing. 
        %        {, <, #, %, $, ' and ": go to next line. 
        %        _, }, ^, &, >, - and ~: stay at end of broken line. 
        % Use of \textquotesingle for straight quote. 
        \newcommand*{\Wrappedbreaksatspecials}{% 
            \def\PYGZus{\discretionary{\char`\_}{\Wrappedafterbreak}{\char`\_}}% 
            \def\PYGZob{\discretionary{}{\Wrappedafterbreak\char`\{}{\char`\{}}% 
            \def\PYGZcb{\discretionary{\char`\}}{\Wrappedafterbreak}{\char`\}}}% 
            \def\PYGZca{\discretionary{\char`\^}{\Wrappedafterbreak}{\char`\^}}% 
            \def\PYGZam{\discretionary{\char`\&}{\Wrappedafterbreak}{\char`\&}}% 
            \def\PYGZlt{\discretionary{}{\Wrappedafterbreak\char`\<}{\char`\<}}% 
            \def\PYGZgt{\discretionary{\char`\>}{\Wrappedafterbreak}{\char`\>}}% 
            \def\PYGZsh{\discretionary{}{\Wrappedafterbreak\char`\#}{\char`\#}}% 
            \def\PYGZpc{\discretionary{}{\Wrappedafterbreak\char`\%}{\char`\%}}% 
            \def\PYGZdl{\discretionary{}{\Wrappedafterbreak\char`\$}{\char`\$}}% 
            \def\PYGZhy{\discretionary{\char`\-}{\Wrappedafterbreak}{\char`\-}}% 
            \def\PYGZsq{\discretionary{}{\Wrappedafterbreak\textquotesingle}{\textquotesingle}}% 
            \def\PYGZdq{\discretionary{}{\Wrappedafterbreak\char`\"}{\char`\"}}% 
            \def\PYGZti{\discretionary{\char`\~}{\Wrappedafterbreak}{\char`\~}}% 
        } 
        % Some characters . , ; ? ! / are not pygmentized. 
        % This macro makes them "active" and they will insert potential linebreaks 
        \newcommand*{\Wrappedbreaksatpunct}{% 
            \lccode`\~`\.\lowercase{\def~}{\discretionary{\hbox{\char`\.}}{\Wrappedafterbreak}{\hbox{\char`\.}}}% 
            \lccode`\~`\,\lowercase{\def~}{\discretionary{\hbox{\char`\,}}{\Wrappedafterbreak}{\hbox{\char`\,}}}% 
            \lccode`\~`\;\lowercase{\def~}{\discretionary{\hbox{\char`\;}}{\Wrappedafterbreak}{\hbox{\char`\;}}}% 
            \lccode`\~`\:\lowercase{\def~}{\discretionary{\hbox{\char`\:}}{\Wrappedafterbreak}{\hbox{\char`\:}}}% 
            \lccode`\~`\?\lowercase{\def~}{\discretionary{\hbox{\char`\?}}{\Wrappedafterbreak}{\hbox{\char`\?}}}% 
            \lccode`\~`\!\lowercase{\def~}{\discretionary{\hbox{\char`\!}}{\Wrappedafterbreak}{\hbox{\char`\!}}}% 
            \lccode`\~`\/\lowercase{\def~}{\discretionary{\hbox{\char`\/}}{\Wrappedafterbreak}{\hbox{\char`\/}}}% 
            \catcode`\.\active
            \catcode`\,\active 
            \catcode`\;\active
            \catcode`\:\active
            \catcode`\?\active
            \catcode`\!\active
            \catcode`\/\active 
            \lccode`\~`\~ 	
        }
    

    \let\OriginalVerbatim=\Verbatim
    
    \renewcommand{\Verbatim}[1][1]{%
        %\parskip\z@skip
        \sbox\Wrappedcontinuationbox {\Wrappedcontinuationsymbol}%
        \sbox\Wrappedvisiblespacebox {\FV@SetupFont\Wrappedvisiblespace}%
        \def\FancyVerbFormatLine ##1{\hsize\linewidth
            \vtop{\raggedright\hyphenpenalty\z@\exhyphenpenalty\z@
                \doublehyphendemerits\z@\finalhyphendemerits\z@
                \strut ##1\strut}%
        }%
        % If the linebreak is at a space, the latter will be displayed as visible
        % space at end of first line, and a continuation symbol starts next line.
        % Stretch/shrink are however usually zero for typewriter font.
        \def\FV@Space {%
            \nobreak\hskip\z@ plus\fontdimen3\font minus\fontdimen4\font
            \discretionary{\copy\Wrappedvisiblespacebox}{\Wrappedafterbreak}
            {\kern\fontdimen2\font}%
        }%
        
        % Allow breaks at special characters using \PYG... macros.
        \Wrappedbreaksatspecials
        % Breaks at punctuation characters . , ; ? ! and / need catcode=\active 	
        \OriginalVerbatim[#1,codes*=\Wrappedbreaksatpunct]%
    }
    

    % Exact colors from NB
    \definecolor{incolor}{HTML}{303F9F}
    \definecolor{outcolor}{HTML}{D84315}
    \definecolor{cellborder}{HTML}{CFCFCF}
    \definecolor{cellbackground}{HTML}{F7F7F7}
    
    % prompt
    
    \newcommand{\boxspacing}{\kern\kvtcb@left@rule\kern\kvtcb@boxsep}
    
    \newcommand{\prompt}[4]{
        {\ttfamily\llap{{\color{#2}[#3]:\hspace{3pt}#4}}\vspace{-\baselineskip}}
    }
    

    
    % Prevent overflowing lines due to hard-to-break entities
    \sloppy 
    % Setup hyperref package
    
    % Slightly bigger margins than the latex defaults
    
    
    
    

\makeatother

\begin{document}
\title{How to use Pandas for SQL-like actions?}
\author{\href{https://www.linkedin.com/in/hesam-mohammad-hosseini/}{Hessam Mohammad Hosseini}}
\maketitle
\begin{abstract}
This document is part 1 of my cheat sheet on \textbf{Pandas} which
provide overall review on how to use \textbf{Pandas} for those familiar
with \textbf{SQL}. 
\end{abstract}
\tableofcontents{}

%\pagebreak\LyXZeroWidthSpace\fancyhf{}               % Clear fancy header/footer 
%\fancyhead[L]{Hessam Mohammad Hosseini}   % My name in Left footer 
%\fancyhead[R]{Pandas for SQL users}  % {\thepage} Page number in Right footer 
%\fancyfoot[L]{Hessam Mohammad Hosseini}   % My name in Left footer 
%\fancyfoot[R]{Pandas for SQL users}  % {\thepage} Page number in Right footer 
%\makeatletter \let\ps@plain\ps@fancy   % Plain page style = fancy page style \makeatother
%
%\fancyhead{} % clear all header fields 
%\fancyhead[RO,LE]{\textbf{Pandas for SQL users}} 
%\fancyfoot{} % clear all footer fields 
%\fancyfoot[LE,RO]{\thepage} 
%\fancyfoot[LO,CE]{Hessam Mohammad Hosseini} 
%\fancyfoot[CO,RE]{\href{https://github.com/jupihes}{jupihes@github}} 
%\renewcommand{\headrulewidth}{0.4pt} 
%\renewcommand{\footrulewidth}{0.4pt}

In this cheetsheet, I try to make readable and easy to use reference
for \href{https://en.wikipedia.org/wiki/SQL}{SQL} users aiming to
explain how to do similar action in \href{https://pandas.pydata.org/}{Pandas}.
My assumption is you know how to have your data as dataframe in Pandas.
As soon as you have a dataframe, you can query like a table in SQL.
Many possibilities are available \href{https://pandas.pydata.org/pandas-docs/stable/user_guide/io.html}{IO Tools (CSV, EXCEL,DB connection \ldots)}but
most important ones are reading from Excel and CSV. 

For example, following line of code will read file located at \texttt{file\_address},
skip first row, consider \texttt{Time} column to be parsed as \texttt{datetime}
format and consider\texttt{ `,`} as thousands identifier while reading
file. \\

\texttt{df\ =\ pd.read\_csv(file\_name, }

\texttt{~~~~~~~~~~~~~~~~~skiprows=10, \# }ignore
first 10 rows of file

\texttt{~~~~~~~~~~~~~~~~~parse\_dates={[}\textquotesingle Time\textquotesingle{]},
\# }Parse \textcolor{green}{Time} as datetime column

\texttt{~~~~~~~~~~~~~~~~~thousands=\textquotesingle ,\textquotesingle )
\# }numbers have , to demonstrate 000 separation\texttt{}~\\

\begin{enumerate}
\item \href{https://pandas.pydata.org/pandas-docs/stable/reference/api/pandas.read_csv.html}{read\_csv}~Function
has lots of useful options to fasilitate work and avoid doing extra
data cleaning tasks after data loading. Here are options for \texttt{read\_csv}.\texttt{ }~\\
\texttt{df\ =\ pandas.read\_csv(filepath\_or\_buffer,\ sep=\textquotesingle ,\ \textquotesingle ,\ delimiter=None,\ header=\textquotesingle infer\textquotesingle ,\ names=None,\ index\_col=None,\ usecols=None,\ squeeze=False,\ prefix=None,\ mangle\_dupe\_cols=True,\ dtype=None,\ engine=None,\ converters=None,\ true\_values=None,\ false\_values=None,\ skipinitialspace=False,\ skiprows=None,\ skipfooter=0,\ nrows=None,\ na\_values=None,\ keep\_default\_na=True,\ na\_filter=True,\ verbose=False,\ skip\_blank\_lines=True,\ parse\_dates=False,\ infer\_datetime\_format=False,\ keep\_date\_col=False,\ date\_parser=None,\ dayfirst=False,\ iterator=False,\ chunksize=None,\ compression=\textquotesingle infer\textquotesingle ,\ thousands=None,\ decimal=b\textquotesingle .\textquotesingle ,\ lineterminator=None,\ quotechar=\textquotesingle "\textquotesingle ,\ quoting=0,\ doublequote=True,\ escapechar=None,\ comment=None,\ encoding=None,\ dialect=None,\ tupleize\_cols=None,\ error\_bad\_lines=True,\ warn\_bad\_lines=True,\ delim\_whitespace=False,\ low\_memory=True,\ memory\_map=False,\ float\_precision=None)} 
\item \href{https://pandas.pydata.org/pandas-docs/stable/reference/api/pandas.read_excel.html\%5C\#pandas.read_excel}{read\_excel}~Functions
has lots of useful options to fasilitate work. Here are options for
\texttt{read\_excel}:\\
 \texttt{df\ =\ pandas.read\_excel(io,\ sheet\_name=0,\ header=0,\ names=None,\ index\_col=None,\ parse\_cols=None,\ usecols=None,\ squeeze=False,\ dtype=None,\ engine=None,\ converters=None,\ true\_values=None,\ false\_values=None,\ skiprows=None,\ nrows=None,\ na\_values=None,\ keep\_default\_na=True,\ verbose=False,\ parse\_dates=False,\ date\_parser=None,\ thousands=None,\ comment=None,\ skip\_footer=0,\ skipfooter=0,\ convert\_float=True,\ mangle\_dupe\_cols=True,\ {*}{*}kwds)} 
\end{enumerate}
I tried to summarize and add to what was available on \href{https://pandas.pydata.org/pandas-docs/stable/getting_started/comparison/comparison_with_sql.html}{Pandas comparison with SQL}
aiming to simplify understanding. For details on functionality, please
check and review \href{http://pandas.pydata.org/pandas-docs/stable}{Pandas documentation}.

\section{SELECT}

\label{select}

\begin{longtable}[c]{|@{}c|l|}
\hline 
\begin{minipage}[b]{0.29\columnwidth}%
\centering SQL Sample\strut %
\end{minipage} &
\begin{minipage}[b]{0.34\columnwidth}%
\centering Pandas Sample\strut %
\end{minipage}\tabularnewline
\endhead
\hline 
\begin{minipage}[t]{0.29\columnwidth}%
\centering \texttt{select\ {*}}

\texttt{FROM\ table}\strut %
\end{minipage} &
\begin{minipage}[t]{0.34\columnwidth}%
\centering \texttt{table}\strut %
\end{minipage}\tabularnewline
\hline 
\begin{minipage}[t]{0.29\columnwidth}%
\centering \texttt{select\ distinct\ c5}

\texttt{FROM\ table}\strut %
\end{minipage} &
\begin{minipage}[t]{0.34\columnwidth}%
\centering \texttt{table{[}\textquotesingle c5\textquotesingle{]}.unique()}
or \texttt{table{[}{[}\textquotesingle c5\textquotesingle{]}{]}.drop\_duplicates()}\strut %
\end{minipage}\tabularnewline
\hline 
\begin{minipage}[t]{0.29\columnwidth}%
\centering \texttt{select\ c1,\ c2,c10}

\texttt{FROM\ table}\strut %
\end{minipage} &
\begin{minipage}[t]{0.34\columnwidth}%
\centering \texttt{df{[}{[}c1,\ c2,c10{]}{]}}\strut %
\end{minipage}\tabularnewline
\hline 
\begin{minipage}[t]{0.29\columnwidth}%
\centering \texttt{select\ c10,\ c2,\ c1}

\texttt{FROM\ table}\strut %
\end{minipage} &
\begin{minipage}[t]{0.34\columnwidth}%
\centering \texttt{df{[}{[}c10,\ c1,\ c2{]}{]}}\strut %
\end{minipage}\tabularnewline
\hline 
\begin{minipage}[t]{0.29\columnwidth}%
\centering \texttt{select\ c10,\ c2{*}12\ -\ c1\ +\ c6}

\texttt{FROM\ table}\strut %
\end{minipage} &
\begin{minipage}[t]{0.34\columnwidth}%
\centering \texttt{df{[}\textquotesingle new\ c\textquotesingle{]}\ =\ df.c2{*}12\ -\ df.c1\ +\ df.c6df{[}{[}c10,\textquotesingle new\ c\textquotesingle{]}{]}}
or \texttt{df.assign(c10\ =\ df.c10,\ new\_c\ =\ df.c2{*}12\ -\ df.c1\ +\ df.c6)}\strut %
\end{minipage}\tabularnewline
\hline 
\begin{minipage}[t]{0.29\columnwidth}%
\centering \texttt{SELECT\ total\_bill,\ tip,\ smoker,\ time}

\texttt{FROM\ tips}

\texttt{LIMIT\ 5}\strut %
\end{minipage} &
\begin{minipage}[t]{0.34\columnwidth}%
\centering \texttt{tips{[}{[}\textquotesingle total\_bill\textquotesingle ,\ \textquotesingle tip\textquotesingle ,\ \textquotesingle smoker\textquotesingle ,\textquotesingle time\textquotesingle{]}{]}.head(5)}\strut %
\end{minipage}\tabularnewline
\hline 
\end{longtable}

\subsection{Update or delete}

\label{update-or-delete}

\begin{longtable}[c]{@{}ccc@{}}
\toprule 
\begin{minipage}[b]{0.29\columnwidth}%
\centering \strut %
\end{minipage} &
\begin{minipage}[b]{0.34\columnwidth}%
\centering SQL Sample\strut %
\end{minipage} &
\begin{minipage}[b]{0.29\columnwidth}%
\centering Pandas Sample\strut %
\end{minipage}\tabularnewline
\endhead
\midrule 
\begin{minipage}[t]{0.29\columnwidth}%
\centering Delete\strut %
\end{minipage} &
\begin{minipage}[t]{0.34\columnwidth}%
\centering \texttt{DELETE\ }

\texttt{FROM\ tips}

\texttt{WHERE\ tip\ \textgreater{}\ 9}\strut %
\end{minipage} &
\begin{minipage}[t]{0.29\columnwidth}%
\centering \texttt{tips\ =\ tips.loc{[}tips{[}\textquotesingle tip\textquotesingle{]}\ \textless{}=\ 9{]}}\strut %
\end{minipage}\tabularnewline
\midrule 
\begin{minipage}[t]{0.29\columnwidth}%
\centering Update\strut %
\end{minipage} &
\begin{minipage}[t]{0.34\columnwidth}%
\centering \texttt{UPDATE\ tips}

\texttt{SET\ tip\ =\ tip{*}2}

\texttt{WHERE\ tip\ \textless{}\ 2}\strut %
\end{minipage} &
\begin{minipage}[t]{0.29\columnwidth}%
\centering \texttt{tips.loc{[}tips{[}\textquotesingle tip\textquotesingle{]}\ \textless{}\ 2,\ \textquotesingle tip\textquotesingle{]}\ {*}=\ 2}\strut %
\end{minipage}\tabularnewline
\bottomrule
\end{longtable}

\section{Conditioning at WHERE}

\label{conditioning-at-where}

\begin{longtable}[c]{@{}ccc@{}}
\toprule 
\begin{minipage}[b]{0.29\columnwidth}%
\centering \strut %
\end{minipage} &
\begin{minipage}[b]{0.34\columnwidth}%
\centering SQL Sample\strut %
\end{minipage} &
\begin{minipage}[b]{0.29\columnwidth}%
\centering Pandas Sample\strut %
\end{minipage}\tabularnewline
\endhead
\midrule 
\begin{minipage}[t]{0.29\columnwidth}%
\centering -\strut %
\end{minipage} &
\begin{minipage}[t]{0.34\columnwidth}%
\centering \texttt{SQL\ SELECT\ {*}}

\texttt{FROM\ tips}

\texttt{WHERE\ time\ =\ \textquotesingle Dinner\textquotesingle}

\texttt{LIMIT\ 5}\strut %
\end{minipage} &
\begin{minipage}[t]{0.29\columnwidth}%
\centering \texttt{tips{[}tips{[}\textquotesingle time\textquotesingle{]}\ ==\ \textquotesingle Dinner\textquotesingle{]}.head(5)}
or

\texttt{is\_dinner\ =\ tips{[}\textquotesingle time\textquotesingle{]}\ ==\ \textquotesingle Dinner\textquotesingle tips{[}is\_dinner{]}.head(5)}\strut %
\end{minipage}\tabularnewline
\midrule 
\begin{minipage}[t]{0.29\columnwidth}%
\centering AND\strut %
\end{minipage} &
\begin{minipage}[t]{0.34\columnwidth}%
\centering \texttt{SELECT\ {*}}

\texttt{FROM\ tips}

\texttt{WHERE\ time\ =\ \textquotesingle Dinner\textquotesingle\ AND\ tip\ \textgreater{}\ 5.00}\strut %
\end{minipage} &
\begin{minipage}[t]{0.29\columnwidth}%
\centering \texttt{tips{[}(tips{[}\textquotesingle time\textquotesingle{]}\ ==\ \textquotesingle Dinner\textquotesingle )\ \&\ (tips{[}\textquotesingle tip\textquotesingle{]}\ \textgreater{}\ 5.00){]}}\strut %
\end{minipage}\tabularnewline
\midrule 
\begin{minipage}[t]{0.29\columnwidth}%
\centering OR\strut %
\end{minipage} &
\begin{minipage}[t]{0.34\columnwidth}%
\centering \texttt{SELECT\ {*}}

\texttt{FROM\ tips}

\texttt{WHERE\ size\ \textgreater{}=\ 5\ OR\ total\_bill\ \textgreater{}\ 45}\strut %
\end{minipage} &
\begin{minipage}[t]{0.29\columnwidth}%
\centering \texttt{tips{[}(tips{[}\textquotesingle size\textquotesingle{]}\ \textgreater{}=\ 5\textquotesingle )\ or\ (tips{[}\textquotesingle total\_bill\textquotesingle{]}\ \textgreater{}\ 45){]}}\strut %
\end{minipage}\tabularnewline
\midrule 
\begin{minipage}[t]{0.29\columnwidth}%
\centering IS NULL\strut %
\end{minipage} &
\begin{minipage}[t]{0.34\columnwidth}%
\centering \texttt{SELECT\ {*}}

\texttt{FROM\ t}

\texttt{WHERE\ col2\ IS\ NULL}\strut %
\end{minipage} &
\begin{minipage}[t]{0.29\columnwidth}%
\centering \texttt{t{[}t{[}\textquotesingle col2\textquotesingle{]}.isna(){]}}\strut %
\end{minipage}\tabularnewline
\midrule 
\begin{minipage}[t]{0.29\columnwidth}%
\centering IS NOT NULL\strut %
\end{minipage} &
\begin{minipage}[t]{0.34\columnwidth}%
\centering \texttt{SELECT\ {*}}

\texttt{FROM\ t}

\texttt{WHERE\ col\ IS\ NOT\ NULL}\strut %
\end{minipage} &
\begin{minipage}[t]{0.29\columnwidth}%
\centering \texttt{t{[}t{[}\textquotesingle col2\textquotesingle{]}.notna(){]}}\strut %
\end{minipage}\tabularnewline
\bottomrule
\end{longtable}

\begin{longtable}[c]{@{}cc}
\toprule 
\begin{minipage}[b]{0.29\columnwidth}%
\centering SQL Sample\strut %
\end{minipage} &
\begin{minipage}[b]{0.34\columnwidth}%
\centering Pandas Sample\strut %
\end{minipage}\tabularnewline
\endhead
\midrule 
\begin{minipage}[t]{0.29\columnwidth}%
\centering \texttt{SELECT\ {*}}

\texttt{FROM\ tips}

\texttt{ORDER\ BY\ tip\ DESC}

\texttt{LIMIT\ 10\ OFFSET\ 5}\strut %
\end{minipage} &
\begin{minipage}[t]{0.34\columnwidth}%
\centering \texttt{tips.nlargest(10\ +\ 5,\ columns=\textquotesingle tip\textquotesingle ).tail(10)}\strut %
\end{minipage}\tabularnewline
\midrule 
\begin{minipage}[t]{0.29\columnwidth}%
\centering \texttt{SELECT\ total\_bill,\ tip,\ smoker,\ time}

\texttt{FROM\ tips}

\texttt{ORDER\ BY\ tip\ }

\texttt{DESC}

\texttt{LIMIT\ 10\ OFFSET\ 5}\strut %
\end{minipage} &
\begin{minipage}[t]{0.34\columnwidth}%
\centering \texttt{tips{[}{[}\textquotesingle total\_bill\textquotesingle ,\ \textquotesingle tip\textquotesingle ,\ \textquotesingle smoker\textquotesingle ,\textquotesingle time\textquotesingle{]}{]}\ tips.nlargest(10\ +\ 5,\ columns=\textquotesingle tip\textquotesingle ).tail(10)}\strut %
\end{minipage}\tabularnewline
\bottomrule
\end{longtable}

\section{GROUP BY}

\label{group-by}

\begin{longtable}[c]{@{}cc}
\toprule 
\begin{minipage}[b]{0.29\columnwidth}%
\centering SQL Sample\strut %
\end{minipage} &
\begin{minipage}[b]{0.34\columnwidth}%
\centering Pandas Sample\strut %
\end{minipage}\tabularnewline
\endhead
\midrule 
\begin{minipage}[t]{0.29\columnwidth}%
\centering \texttt{SELECT\ sex,\ count({*})}

\texttt{FROM\ tips}

\texttt{GROUP\ BY\ sex}\strut %
\end{minipage} &
\begin{minipage}[t]{0.34\columnwidth}%
\centering \texttt{tips.groupby(\textquotesingle sex\textquotesingle ).size()}

or

\texttt{tips.groupby(\textquotesingle sex\textquotesingle ){[}\textquotesingle total\_bill\textquotesingle{]}.count()}\strut %
\end{minipage}\tabularnewline
\begin{minipage}[t]{0.29\columnwidth}%
\centering \texttt{select\ A,\ sum(C),\ sum(D)}

\texttt{FROM\ df}

\texttt{GROUP\ BY\ A}\strut %
\end{minipage} &
\begin{minipage}[t]{0.34\columnwidth}%
\centering \texttt{df.groupby(\textquotesingle A\textquotesingle ){[}\textquotesingle B\textquotesingle ,\textquotesingle C\textquotesingle{]}.sum()}\strut %
\end{minipage}\tabularnewline
\bottomrule
\end{longtable}

\begin{longtable}[c]{@{}cc}
\toprule 
\begin{minipage}[b]{0.29\columnwidth}%
\centering SQL Sample\strut %
\end{minipage} &
\begin{minipage}[b]{0.34\columnwidth}%
\centering Pandas Sample\strut %
\end{minipage}\tabularnewline
\endhead
\midrule 
\begin{minipage}[t]{0.29\columnwidth}%
\centering \texttt{SELECT\ day,\ AVG(tip),\ COUNT({*})}

\texttt{FROM\ tips}

\texttt{GROUP\ BY\ day}\strut %
\end{minipage} &
\begin{minipage}[t]{0.34\columnwidth}%
\centering \texttt{tips.groupby(\textquotesingle day\textquotesingle ).agg(\{\textquotesingle tip\textquotesingle :\ np.mean,\ \textquotesingle day\textquotesingle :\ np.size\})}\strut %
\end{minipage}\tabularnewline
\bottomrule
\end{longtable}

\begin{longtable}[c]{@{}cc}
\toprule 
\begin{minipage}[b]{0.29\columnwidth}%
\centering SQL Sample\strut %
\end{minipage} &
\begin{minipage}[b]{0.34\columnwidth}%
\centering Pandas Sample\strut %
\end{minipage}\tabularnewline
\endhead
\midrule 
\begin{minipage}[t]{0.29\columnwidth}%
\centering \texttt{SELECT\ smoker,\ day,\ COUNT({*}),\ AVG(tip)}

\texttt{FROM\ tips}

\texttt{GROUP\ BY\ smoker,\ day}\strut %
\end{minipage} &
\begin{minipage}[t]{0.34\columnwidth}%
\centering \texttt{tips.groupby({[}\textquotesingle smoker,\textquotesingle day\textquotesingle{]}).agg(\{\textquotesingle tip\textquotesingle :\ {[}np.size,\ np.mean{]}\})}\strut %
\end{minipage}\tabularnewline
\bottomrule
\end{longtable}

\begin{longtable}[c]{@{}cc}
\toprule 
\begin{minipage}[b]{0.29\columnwidth}%
\centering SQL Sample\strut %
\end{minipage} &
\begin{minipage}[b]{0.34\columnwidth}%
\centering Pandas Sample\strut %
\end{minipage}\tabularnewline
\endhead
\midrule 
\begin{minipage}[t]{0.29\columnwidth}%
\centering \texttt{SELECT\ c1,\ COUNT({*})}

\texttt{FROM\ df}

\texttt{where\ country=\textquotesingle IR\textquotesingle}

\texttt{GROUP\ BY\ c1}

\texttt{having\ count({*})\textgreater{}1000}\strut %
\end{minipage} &
\begin{minipage}[t]{0.34\columnwidth}%
\centering \texttt{df{[}df.country\ ==\ \textquotesingle IR\textquotesingle{]}.groupby(\textquotesingle c1\textquotesingle ).filter(lambda\ g:\ len(g)\ \textgreater{}\ 1000).groupby(\textquotesingle c1\textquotesingle ).size()}\strut %
\end{minipage}\tabularnewline
\begin{minipage}[t]{0.29\columnwidth}%
\centering \texttt{SELECT\ c1,\ COUNT({*})}

\texttt{FROM\ df}

\texttt{WHERE\ country=\textquotesingle IR\textquotesingle}

\texttt{GROUP\ BY\ c1}

\texttt{HAVING\ count({*})\textgreater{}1000}

\texttt{ORDER\ BY\ count({*})\ desc}\strut %
\end{minipage} &
\begin{minipage}[t]{0.34\columnwidth}%
\centering \texttt{df{[}df.country\ ==\ \textquotesingle IR\textquotesingle{]}}

\texttt{.groupby(\textquotesingle c1\textquotesingle ).}

\texttt{filter(lambda\ g:\ len(g)\ \textgreater{}\ 1000)}

\texttt{.groupby(\textquotesingle c1\textquotesingle ).size()}

\texttt{.sort\_values(ascending=False)}\strut %
\end{minipage}\tabularnewline
\bottomrule
\end{longtable}

\section{ORDER BY}

\label{order-by}

\begin{longtable}[c]{@{}cc}
\toprule 
\begin{minipage}[b]{0.29\columnwidth}%
\centering SQL Sample\strut %
\end{minipage} &
\begin{minipage}[b]{0.34\columnwidth}%
\centering Pandas Sample\strut %
\end{minipage}\tabularnewline
\endhead
\midrule 
\begin{minipage}[t]{0.29\columnwidth}%
\centering \texttt{SELECT\ {*}}

\texttt{FROM\ df}

\texttt{ORDER\ BY\ A,\ B}\strut %
\end{minipage} &
\begin{minipage}[t]{0.34\columnwidth}%
\centering \texttt{df.sort\_values({[}\textquotesingle A\textquotesingle ,\ \textquotesingle B\textquotesingle{]})}\strut %
\end{minipage}\tabularnewline
\begin{minipage}[t]{0.29\columnwidth}%
\centering \texttt{SELECT\ {*}}

\texttt{FROM\ df}

\texttt{ORDER\ BY\ A\ desc,\ C}\strut %
\end{minipage} &
\begin{minipage}[t]{0.34\columnwidth}%
\centering \texttt{df.sort\_values({[}\textquotesingle A\textquotesingle ,\ \textquotesingle B\textquotesingle{]},ascending={[}False,\ True{]})}\strut %
\end{minipage}\tabularnewline
\bottomrule
\end{longtable}

\section{UNION, JOIN and other set related operations}

\label{union-join-and-other-set-related-operations}

I will work to provide more comprehensive explanations on this part.

\subsection{Union}

\label{union}

\begin{longtable}[c]{|@{}c|l|}
\hline 
\begin{minipage}[b]{0.29\columnwidth}%
\centering SQL Sample\strut %
\end{minipage} &
\begin{minipage}[b]{0.34\columnwidth}%
\centering Pandas Sample\strut %
\end{minipage}\tabularnewline
\endhead
\hline 
\begin{minipage}[t]{0.29\columnwidth}%
\centering \texttt{SELECT\ c1,\ c2}

\texttt{FROM\ df1}

\texttt{UNION\ ALL}

\texttt{SELECT\ c1,\ c2}

\texttt{FROM\ df2}\strut %
\end{minipage} &
\begin{minipage}[t]{0.34\columnwidth}%
\centering \texttt{pd.concat({[}df1,\ df2{]})}\strut %
\end{minipage}\tabularnewline
\hline 
\end{longtable}

Difference between \texttt{union\ all} and \texttt{union} is that
\texttt{union} will remove duplicates.

\begin{longtable}[c]{|@{}c|l|}
\hline 
\begin{minipage}[b]{0.29\columnwidth}%
\centering SQL Sample\strut %
\end{minipage} &
\begin{minipage}[b]{0.34\columnwidth}%
\centering Pandas Sample\strut %
\end{minipage}\tabularnewline
\endhead
\hline 
\begin{minipage}[t]{0.29\columnwidth}%
\centering \texttt{SELECT\ c1,\ c2}

\texttt{FROM\ df1}

\texttt{UNION}

\texttt{SELECT\ c1,\ c2}

\texttt{FROM\ df2}\strut %
\end{minipage} &
\begin{minipage}[t]{0.34\columnwidth}%
\centering \texttt{pd.concat({[}df1,\ df2{]})}

\texttt{.drop\_duplicates()}\strut %
\end{minipage}\tabularnewline
\hline 
\end{longtable}

\subsection{Different Join cases}

\label{different-join-cases}

I will add parts to make explanation on \texttt{join} more clear and
comprehensive. Below image extracted from \href{https://www.enthought.com/}{Enthought}named
``Enthought-Python-Pandas-Cheat-Sheets-1-8-v1.0.2'' worth more than
100 sentences to explain different types of \texttt{join}. You can
get whole file via registration on \href{https://www.enthought.com/}{Enthought}.
\includegraphics{8D__Hesam_Data_Science_My_Summaries_-_Cheat_Sheets_SQL-versus-Pandas-master_Join.png}

\begin{longtable}[c]{|@{}c|l|}
\hline 
\begin{minipage}[b]{0.29\columnwidth}%
\centering SQL Sample\strut %
\end{minipage} &
\begin{minipage}[b]{0.34\columnwidth}%
\centering Pandas Sample\strut %
\end{minipage}\tabularnewline
\endhead
\hline 
\begin{minipage}[t]{0.29\columnwidth}%
\centering \texttt{SELECT\ {*}}

\texttt{FROM\ df1}

\texttt{INNER\ JOIN\ df2}

\texttt{ON\ df1.key\ =\ df2.key}\strut %
\end{minipage} &
\begin{minipage}[t]{0.34\columnwidth}%
\centering \texttt{pd.merge(df1,\ df2,\ on=\textquotesingle key\textquotesingle )}\strut %
\end{minipage}\tabularnewline
\hline 
\begin{minipage}[t]{0.29\columnwidth}%
\centering \texttt{SELECT\ {*}}

\texttt{FROM\ df1}

\texttt{INNER\ JOIN\ df2}

\texttt{ON\ df1.c7\ =\ df2.c5}\strut %
\end{minipage} &
\begin{minipage}[t]{0.34\columnwidth}%
\centering \texttt{pd.merge(df1,\ df2,\ left\_on=\textquotesingle c7\textquotesingle ,right\_on=\textquotesingle c5\textquotesingle )}\strut %
\end{minipage}\tabularnewline
\hline 
\end{longtable}

\section{Time functionality}

\label{time-functionality}

In order to have possibility to use time related functionalities,
we need help \texttt{Pandas} understand which columns are to be treated
as time. Of course, the columns should be in for that converting them
to time format is possible. For details, please check and review \href{http://pandas.pydata.org/pandas-docs/stable}{Pandas documentation}.
If you manage to let \texttt{Pandas} know properly which column(s)
to be time related column(s), they will end up having \texttt{datetime64{[}ns{]}}
format. \texttt{.dtypes} on dataframe provides you with columns formats.
Pass date related column(s) you need to \texttt{parse\_dates} to \texttt{read\_csv}
or \texttt{read\_excel} functions. Check Pandas documentation for
more details.

Doing so, you can apply \texttt{.dt} on column to have date - time
selection like 

\texttt{dt.dayofweek}

\texttt{dt.minute}

\texttt{dt.hour}

\texttt{dt.second}

\texttt{dt.quarter}

\texttt{dt.month}

\texttt{dt.month\_name}

\texttt{dt.weekday\_name}

\texttt{dt.weekday}

\texttt{dt.weekofyear}

\texttt{dt.year}

\subsection{How to get current date time using pandas?}

\label{how-to-get-current-date-time-using-pandas}

\texttt{pd.datetime.now()} 

\texttt{pd.datetime.now().date()}

\texttt{pd.datetime.now().year}

\texttt{pd.datetime.now().month} 

\texttt{pd.datetime.now().day}

\texttt{pd.datetime.now().hour}

\texttt{pd.datetime.now().minute}

\texttt{pd.datetime.now().second}

\texttt{pd.datetime.now().microsecond}

Again, check Pandas documentation for more! Here, we assume \texttt{sdate}
column to have \texttt{datetime64{[}ns{]}} format.

\begin{longtable}[c]{|@{}c|l|c@{}|}
\hline 
\begin{minipage}[b]{0.29\columnwidth}%
\centering \strut %
\end{minipage} &
\begin{minipage}[b]{0.34\columnwidth}%
\centering SQL Sample\strut %
\end{minipage} &
\begin{minipage}[b]{0.29\columnwidth}%
\centering Pandas Sample\strut %
\end{minipage}\tabularnewline
\endhead
\hline 
\begin{minipage}[t]{0.29\columnwidth}%
\centering sysdate - n\strut %
\end{minipage} &
\begin{minipage}[t]{0.34\columnwidth}%
\centering \texttt{SELECT\ {*}}

\texttt{FROM\ df}

\texttt{WHERE\ sdate\textgreater{}\ sysdate-5}\strut %
\end{minipage} &
\begin{minipage}[t]{0.29\columnwidth}%
\centering \texttt{df{[}df{[}\textquotesingle sdate\textquotesingle{]}.dt.date()\textgreater{}\ \ pd.datetime.now().date()-5{]}}\strut %
\end{minipage}\tabularnewline
\hline 
\begin{minipage}[t]{0.29\columnwidth}%
\centering month\strut %
\end{minipage} &
\begin{minipage}[t]{0.34\columnwidth}%
\centering \texttt{SELECT\ {*}}

\texttt{FROM\ df}

\texttt{WHERE\ sdate\ in\ Q1}\strut %
\end{minipage} &
\begin{minipage}[t]{0.29\columnwidth}%
\centering \texttt{df{[}(df.sdate.dt.month\ \textgreater{}=\ 1)\ \&\ (df.sdate.dt.month\ \textless{}=\ 3){]}}\strut %
\end{minipage}\tabularnewline
\hline 
\begin{minipage}[t]{0.29\columnwidth}%
\centering between\strut %
\end{minipage} &
\begin{minipage}[t]{0.34\columnwidth}%
\centering \texttt{SELECT\ {*}}

\texttt{FROM\ t}

\texttt{WHERE\ to\_char(sdate,\textquotesingle yyyy\textquotesingle )\ between\ 1998\ AND\ 2018}\strut %
\end{minipage} &
\begin{minipage}[t]{0.29\columnwidth}%
\centering \texttt{t{[}(t.sdate.dt.year\ \textgreater{}=\ 1998)\ \&\ (t.sdate.dt.year\ \textless{}=\ 2018){]}}\strut %
\end{minipage}\tabularnewline
\hline 
\begin{minipage}[t]{0.29\columnwidth}%
\centering \strut %
\end{minipage} &
\begin{minipage}[t]{0.34\columnwidth}%
\centering \texttt{SELECT\ {*}}

\texttt{FROM\ t}

\texttt{WHERE\ to\_char(sdate\ ,\textquotesingle day\textquotesingle )=\ \textquotesingle Friday\textquotesingle}\strut %
\end{minipage} &
\begin{minipage}[t]{0.29\columnwidth}%
\centering \texttt{df{[}df.sdate.dt.day\_name()\ ==\ \textquotesingle Friday\textquotesingle{]}}\strut %
\end{minipage}\tabularnewline
\hline 
\end{longtable}

\section{String related functionality like \texttt{like}, \texttt{Substr}}

For columns with \texttt{string} content, we could access string related
functionality by applying \texttt{.str} on column. Here are few samples:
\texttt{str.contains} - \href{https://pandas.pydata.org/pandas-docs/stable/reference/api/pandas.Series.str.contains.html?}{contains}
options:

\texttt{}%
\noindent\begin{minipage}[t]{1\columnwidth}%
\texttt{Series.str.contains(pat,\ case=True,\ flags=0,\ na=nan,\ regex=True)}%
\end{minipage}

\texttt{Here are list of main `string` functions. }

\texttt{str.upper}

\texttt{str.lower}

\texttt{str.extract}

\texttt{str.extractall}

\texttt{str.find}

\texttt{str.findall}

\texttt{str.len}

\texttt{str.replace}

\texttt{str.slice}

\texttt{str.split}

\texttt{str.strip}

Check \href{https://pandas.pydata.org/pandas-docs/stable/search.html?q=.str.\%5C&check_keywords=yes\%5C&area=default}{Pandas documentation}
for more!

\begin{longtable}[c]{|@{}c|l|c@{}|}
\hline 
\begin{minipage}[b]{0.29\columnwidth}%
\centering \strut %
\end{minipage} &
\begin{minipage}[b]{0.34\columnwidth}%
\centering SQL Sample\strut %
\end{minipage} &
\begin{minipage}[b]{0.29\columnwidth}%
\centering Pandas Sample\strut %
\end{minipage}\tabularnewline
\hline 
\begin{minipage}[t]{0.29\columnwidth}%
\centering regex\strut %
\end{minipage} &
\begin{minipage}[t]{0.34\columnwidth}%
\centering \texttt{SELECT\ upper(trim(to\_char(LAC,\textquotesingle xxxxx\textquotesingle ))\ ||\textquotesingle -\textquotesingle\ ||\ trim(to\_char(CI,\textquotesingle xxxxx\textquotesingle )))\ AS\ "LAC-CI(HEX)"}

\texttt{FROM\ t}\strut %
\end{minipage} &
\begin{minipage}[t]{0.29\columnwidth}%
\centering \texttt{t\ =\ t{[}\textquotesingle LAC\textquotesingle ,\textquotesingle CI\textquotesingle{]}\textbackslash}

\texttt{.apply(lambda\ x:\ x\textbackslash}

\texttt{.astype(str)\textbackslash}

\texttt{.map(lambda\ x:\ int(x,\ base=16)))t}

\texttt{.assig(LAC-CI(HEX)\ =\ t{[}\textquotesingle LAC\textquotesingle{]}+\textquotesingle -\textquotesingle +t{[}\textquotesingle CI\textquotesingle{]}}\strut %
\end{minipage}\tabularnewline
\hline 
\endhead
\hline 
\begin{minipage}[t]{0.29\columnwidth}%
\centering substr\strut %
\end{minipage} &
\begin{minipage}[t]{0.34\columnwidth}%
\centering \texttt{SELECT\ {*}}

\texttt{FROM\ tips}

\texttt{WHERE\ substr(time,1,2)\ \ =\ \textquotesingle Di\textquotesingle}\strut %
\end{minipage} &
\begin{minipage}[t]{0.29\columnwidth}%
\centering \texttt{tips{[}tips{[}\textquotesingle time\textquotesingle{]}\textbackslash}

\texttt{.str{[}:2{]}\ ==\ \textquotesingle Di\textquotesingle{]}}\strut %
\end{minipage}\tabularnewline
\hline 
\begin{minipage}[t]{0.29\columnwidth}%
\centering like\strut %
\end{minipage} &
\begin{minipage}[t]{0.34\columnwidth}%
\centering \texttt{SELECT\ {*}}

\texttt{FROM\ df}

\texttt{WHERE\ Country\ \ like\ \textquotesingle\%IR\%\textquotesingle}\strut %
\end{minipage} &
\begin{minipage}[t]{0.29\columnwidth}%
\centering \texttt{df{[}df{[}\textquotesingle Country\textquotesingle{]}\textbackslash}

\texttt{.str.contains(\textquotesingle IR\textquotesingle )\ ==\ True{]}}\strut %
\end{minipage}\tabularnewline
\hline 
\begin{minipage}[t]{0.29\columnwidth}%
\centering like\strut %
\end{minipage} &
\begin{minipage}[t]{0.34\columnwidth}%
\centering \texttt{SELECT\ {*}}

\texttt{FROM\ df}

\texttt{WHERE\ Country\ \ like\ \textquotesingle IR\%\textquotesingle}\strut %
\end{minipage} &
\begin{minipage}[t]{0.29\columnwidth}%
\centering \texttt{df{[}df{[}\textquotesingle Country\textquotesingle{]}\textbackslash}

\texttt{.str.startswith(\textquotesingle IR\textquotesingle )\ ==\ True{]}}\strut %
\end{minipage}\tabularnewline
\hline 
\begin{minipage}[t]{0.29\columnwidth}%
\centering like\strut %
\end{minipage} &
\begin{minipage}[t]{0.34\columnwidth}%
\centering \texttt{SELECT\ {*}}

\texttt{FROM\ df}

\texttt{WHERE\ Country\ \ like\ \textquotesingle\%AN\textquotesingle}\strut %
\end{minipage} &
\begin{minipage}[t]{0.29\columnwidth}%
\centering \texttt{df{[}df{[}\textquotesingle Country\textquotesingle{]}\textbackslash}

\texttt{.str.endswith(\textquotesingle AN\textquotesingle )\ ==\ True{]}}\strut %
\end{minipage}\tabularnewline
\hline 
\begin{minipage}[t]{0.29\columnwidth}%
\centering in\strut %
\end{minipage} &
\begin{minipage}[t]{0.34\columnwidth}%
\centering \texttt{SELECT\ {*}}

\texttt{FROM\ df}

\texttt{WHERE\ City\ in\ (\textquotesingle TEHRAN\textquotesingle ,\ \textquotesingle BERLIN\textquotesingle ,\textquotesingle STOKHOLM\textquotesingle )}\strut %
\end{minipage} &
\begin{minipage}[t]{0.29\columnwidth}%
\centering \texttt{df{[}df{[}\textquotesingle City\textquotesingle{]}\textbackslash}

\texttt{.isin({[}\textquotesingle TEHRAN\textquotesingle ,\ \textquotesingle BERLIN\textquotesingle ,\textquotesingle STOKHOLM\textquotesingle{]})}\strut %
\end{minipage}\tabularnewline
\hline 
\begin{minipage}[t]{0.29\columnwidth}%
\centering regex\strut %
\end{minipage} &
\begin{minipage}[t]{0.34\columnwidth}%
\centering \texttt{SELECT\ last\_name}

\texttt{FROM\ contacts}

\texttt{WHERE\ REGEXP\_LIKE\ (last\_name,\ \textquotesingle\^{\ensuremath{+}}A({*})\textquotesingle )}\strut %
\end{minipage} &
\begin{minipage}[t]{0.29\columnwidth}%
\centering \texttt{contacts{[}contacts{[}\textquotesingle last\_name\textquotesingle{]}\textbackslash}

\texttt{.str.contains(\textquotesingle\^{\ensuremath{+}}A({*})\textquotesingle ){]}}\strut %
\end{minipage}\tabularnewline
\hline 
\begin{minipage}[t]{0.29\columnwidth}%
\centering regex\strut %
\end{minipage} &
\begin{minipage}[t]{0.34\columnwidth}%
\centering \texttt{SELECT\ c1}

\texttt{FROM\ t}

\texttt{WHERE\ REGEXP\_LIKE(c1,\textquotesingle ({[}A-Z{]}{[}\textbackslash d{]}\{4\})\textquotesingle )}\strut %
\end{minipage} &
\begin{minipage}[t]{0.29\columnwidth}%
\centering \texttt{t{[}t{[}\textquotesingle c1\textquotesingle{]}\textbackslash}

\texttt{.str.contains(({[}A-Z{]}{[}\textbackslash d{]}\{4\})){]}}\strut %
\end{minipage}\tabularnewline
\end{longtable}

\texttt{\textbar{}\textbar{}} provide concatenation functionality
in PL/SQL. In \textbf{Python}, + on \texttt{string} values resulted
in concatenated text. 

\subparagraph{Oracle's ROW\_NUMBER() analytic function}

\label{oracles-row_number-analytic-function}

\begin{longtable}[c]{@{}cc}
\toprule 
\begin{minipage}[b]{0.29\columnwidth}%
\centering SQL Sample\strut %
\end{minipage} &
\begin{minipage}[b]{0.34\columnwidth}%
\centering Pandas Sample\strut %
\end{minipage}\tabularnewline
\endhead
\midrule 
\begin{minipage}[t]{0.29\columnwidth}%
\centering \texttt{SELECT\ {*}\ }

\texttt{FROM\ }

\texttt{(SELECT\ t.{*}, ROW\_NUMBER()\ OVER(PARTITION\ BY\ day\ ORDER\ BY\ total\_bill\ DESC)\ AS\ r}

\texttt{FROM\ tips\ t)}

\texttt{WHERE\ r\ \textless{}\ }

\texttt{ORDER\ BY\ day,\ r;}\strut %
\end{minipage} &
\begin{minipage}[t]{0.34\columnwidth}%
\centering \texttt{(tips.assign(r=tips\textbackslash}

\texttt{.sort\_values({[}\textquotesingle total\_bill\textquotesingle{]},
ascending=False)\textbackslash}

\texttt{.groupby({[}\textquotesingle day\textquotesingle{]})\textbackslash}

\texttt{.cumcount()+1)\textbackslash}

\texttt{.query(\textquotesingle r\ \textless{}\ 3\textquotesingle )}

\texttt{.sort\_values({[}\textquotesingle day\textquotesingle ,\ \textquotesingle r\textquotesingle{]}))}\strut %
\end{minipage}\tabularnewline
\bottomrule
\end{longtable}

\section{To check for missing values }
\begin{itemize}
\item \texttt{df.notnull()} Use to Drop Missing Values 
\item \texttt{df.dropna()} Filling Missing Values --- Direct Replace
\item \texttt{df.fillna()} 
\end{itemize}
Besides \href{https://pandas.pydata.org/pandas-docs/stable/getting_started/comparison/comparison_with_sql.html}{Pandas comparison with SQL},
I also get ideas from following references: 

1. \href{https://hackernoon.com/pandas-cheatsheet-for-sql-people-part-1-2976894acd0}{pandas-cheatsheet-for-sql-people-part-1}

2. \href{https://medium.com/jbennetcodes/how-to-rewrite-your-sql-queries-in-pandas-and-more-149d341fc53e}{how-to-rewrite-your-sql-queries-in-pandas-and-more}

3. \href{https://www.kaggle.com/anagharumade/thinking-like-sql-in-pandas}{thinking-like-sql-in-pandas}

4. \href{https://medium.com/fintechexplained/did-you-know-pandas-can-do-so-much-f65dc7db3051}{did-you-know-pandas-can-do-so-much}

5. \href{https://towardsdatascience.com/10-python-pandas-tricks-that-make-your-work-more-efficient-2e8e483808ba}{10-python-pandas-tricks-that-make-your-work-more-efficient}

% Add a bibliography block to the postdoc

\end{document}
